\documentclass[tikz]{standalone}
\usepackage{tikz,pgf}
\usetikzlibrary{shapes,fadings}
\begin{document}


\foreach \y in {1,.95,...,0}
	{
	\begin{tikzpicture}
	\pgfmathsetmacro{\z}{100*\y-10}
	\pgfmathsetmacro{\x}{.7*\y}

	\tikzfading [name=radialfade, inner color=transparent!\z, outer color=transparent!100]

	\fill [blue, path fading=radialfade] circle (3);

	\shade[ball color = blue, opacity = \x] (0,0) circle (.2cm); %this is the problem

	%shade[ball color = blue, opacity = \y] (.25,0) circle (0.025cm); %opacity is 0 to 1
	%shade[ball color = blue, opacity = \y] (-.25,0) circle (.25cm); %opacity is 0 to 1
	%shade[ball color = blue, opacity = \y] (0,.25) circle (.25cm); %opacity is 0 to 1
	%shade[ball color = blue, opacity = \y] (0,-.25) circle (.25cm); %opacity is 0 to 1









	\end{tikzpicture}
	}
\foreach \y in {0,.05,...,1}
	{
	\begin{tikzpicture}
	\pgfmathsetmacro{\z}{100*\y-10}
	\pgfmathsetmacro{\x}{.7*\y}

	\tikzfading [name=radialfade, inner color=transparent!\z, outer color=transparent!100]

	\fill [blue, path fading=radialfade] circle (3);

	\shade[ball color = blue, opacity = \x] (0,0) circle (.2cm); %this is the problem

	%shade[ball color = blue, opacity = \y] (.25,0) circle (0.025cm); %opacity is 0 to 1
	%shade[ball color = blue, opacity = \y] (-.25,0) circle (.25cm); %opacity is 0 to 1
	%shade[ball color = blue, opacity = \y] (0,.25) circle (.25cm); %opacity is 0 to 1
	%shade[ball color = blue, opacity = \y] (0,-.25) circle (.25cm); %opacity is 0 to 1









	\end{tikzpicture}
	}



\end{document}
